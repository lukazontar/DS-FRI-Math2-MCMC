% Options for packages loaded elsewhere
\PassOptionsToPackage{unicode}{hyperref}
\PassOptionsToPackage{hyphens}{url}
%
\documentclass[
]{article}
\title{Part 3 - MCMC: Homework}
\author{}
\date{\vspace{-2.5em}}

\usepackage{amsmath,amssymb}
\usepackage{lmodern}
\usepackage{iftex}
\ifPDFTeX
  \usepackage[T1]{fontenc}
  \usepackage[utf8]{inputenc}
  \usepackage{textcomp} % provide euro and other symbols
\else % if luatex or xetex
  \usepackage{unicode-math}
  \defaultfontfeatures{Scale=MatchLowercase}
  \defaultfontfeatures[\rmfamily]{Ligatures=TeX,Scale=1}
\fi
% Use upquote if available, for straight quotes in verbatim environments
\IfFileExists{upquote.sty}{\usepackage{upquote}}{}
\IfFileExists{microtype.sty}{% use microtype if available
  \usepackage[]{microtype}
  \UseMicrotypeSet[protrusion]{basicmath} % disable protrusion for tt fonts
}{}
\makeatletter
\@ifundefined{KOMAClassName}{% if non-KOMA class
  \IfFileExists{parskip.sty}{%
    \usepackage{parskip}
  }{% else
    \setlength{\parindent}{0pt}
    \setlength{\parskip}{6pt plus 2pt minus 1pt}}
}{% if KOMA class
  \KOMAoptions{parskip=half}}
\makeatother
\usepackage{xcolor}
\IfFileExists{xurl.sty}{\usepackage{xurl}}{} % add URL line breaks if available
\IfFileExists{bookmark.sty}{\usepackage{bookmark}}{\usepackage{hyperref}}
\hypersetup{
  pdftitle={Part 3 - MCMC: Homework},
  hidelinks,
  pdfcreator={LaTeX via pandoc}}
\urlstyle{same} % disable monospaced font for URLs
\usepackage[margin=1in]{geometry}
\usepackage{color}
\usepackage{fancyvrb}
\newcommand{\VerbBar}{|}
\newcommand{\VERB}{\Verb[commandchars=\\\{\}]}
\DefineVerbatimEnvironment{Highlighting}{Verbatim}{commandchars=\\\{\}}
% Add ',fontsize=\small' for more characters per line
\usepackage{framed}
\definecolor{shadecolor}{RGB}{248,248,248}
\newenvironment{Shaded}{\begin{snugshade}}{\end{snugshade}}
\newcommand{\AlertTok}[1]{\textcolor[rgb]{0.94,0.16,0.16}{#1}}
\newcommand{\AnnotationTok}[1]{\textcolor[rgb]{0.56,0.35,0.01}{\textbf{\textit{#1}}}}
\newcommand{\AttributeTok}[1]{\textcolor[rgb]{0.77,0.63,0.00}{#1}}
\newcommand{\BaseNTok}[1]{\textcolor[rgb]{0.00,0.00,0.81}{#1}}
\newcommand{\BuiltInTok}[1]{#1}
\newcommand{\CharTok}[1]{\textcolor[rgb]{0.31,0.60,0.02}{#1}}
\newcommand{\CommentTok}[1]{\textcolor[rgb]{0.56,0.35,0.01}{\textit{#1}}}
\newcommand{\CommentVarTok}[1]{\textcolor[rgb]{0.56,0.35,0.01}{\textbf{\textit{#1}}}}
\newcommand{\ConstantTok}[1]{\textcolor[rgb]{0.00,0.00,0.00}{#1}}
\newcommand{\ControlFlowTok}[1]{\textcolor[rgb]{0.13,0.29,0.53}{\textbf{#1}}}
\newcommand{\DataTypeTok}[1]{\textcolor[rgb]{0.13,0.29,0.53}{#1}}
\newcommand{\DecValTok}[1]{\textcolor[rgb]{0.00,0.00,0.81}{#1}}
\newcommand{\DocumentationTok}[1]{\textcolor[rgb]{0.56,0.35,0.01}{\textbf{\textit{#1}}}}
\newcommand{\ErrorTok}[1]{\textcolor[rgb]{0.64,0.00,0.00}{\textbf{#1}}}
\newcommand{\ExtensionTok}[1]{#1}
\newcommand{\FloatTok}[1]{\textcolor[rgb]{0.00,0.00,0.81}{#1}}
\newcommand{\FunctionTok}[1]{\textcolor[rgb]{0.00,0.00,0.00}{#1}}
\newcommand{\ImportTok}[1]{#1}
\newcommand{\InformationTok}[1]{\textcolor[rgb]{0.56,0.35,0.01}{\textbf{\textit{#1}}}}
\newcommand{\KeywordTok}[1]{\textcolor[rgb]{0.13,0.29,0.53}{\textbf{#1}}}
\newcommand{\NormalTok}[1]{#1}
\newcommand{\OperatorTok}[1]{\textcolor[rgb]{0.81,0.36,0.00}{\textbf{#1}}}
\newcommand{\OtherTok}[1]{\textcolor[rgb]{0.56,0.35,0.01}{#1}}
\newcommand{\PreprocessorTok}[1]{\textcolor[rgb]{0.56,0.35,0.01}{\textit{#1}}}
\newcommand{\RegionMarkerTok}[1]{#1}
\newcommand{\SpecialCharTok}[1]{\textcolor[rgb]{0.00,0.00,0.00}{#1}}
\newcommand{\SpecialStringTok}[1]{\textcolor[rgb]{0.31,0.60,0.02}{#1}}
\newcommand{\StringTok}[1]{\textcolor[rgb]{0.31,0.60,0.02}{#1}}
\newcommand{\VariableTok}[1]{\textcolor[rgb]{0.00,0.00,0.00}{#1}}
\newcommand{\VerbatimStringTok}[1]{\textcolor[rgb]{0.31,0.60,0.02}{#1}}
\newcommand{\WarningTok}[1]{\textcolor[rgb]{0.56,0.35,0.01}{\textbf{\textit{#1}}}}
\usepackage{graphicx}
\makeatletter
\def\maxwidth{\ifdim\Gin@nat@width>\linewidth\linewidth\else\Gin@nat@width\fi}
\def\maxheight{\ifdim\Gin@nat@height>\textheight\textheight\else\Gin@nat@height\fi}
\makeatother
% Scale images if necessary, so that they will not overflow the page
% margins by default, and it is still possible to overwrite the defaults
% using explicit options in \includegraphics[width, height, ...]{}
\setkeys{Gin}{width=\maxwidth,height=\maxheight,keepaspectratio}
% Set default figure placement to htbp
\makeatletter
\def\fps@figure{htbp}
\makeatother
\setlength{\emergencystretch}{3em} % prevent overfull lines
\providecommand{\tightlist}{%
  \setlength{\itemsep}{0pt}\setlength{\parskip}{0pt}}
\setcounter{secnumdepth}{-\maxdimen} % remove section numbering
\ifLuaTeX
  \usepackage{selnolig}  % disable illegal ligatures
\fi

\begin{document}
\maketitle

\hypertarget{solution}{%
\subsection{Solution}\label{solution}}

\hypertarget{metropolis-hastings}{%
\subsubsection{Metropolis-Hastings}\label{metropolis-hastings}}

\textbf{Implement Metropolis-Hastings with Multivariate Normal proposal
(mean 0, covariance matrix is a parameter).}

\begin{Shaded}
\begin{Highlighting}[]
\NormalTok{metropolis\_hastings\_mv\_norm }\OtherTok{\textless{}{-}} \ControlFlowTok{function}\NormalTok{(cov, }\AttributeTok{mean=}\DecValTok{0}\NormalTok{) \{}
  
\NormalTok{\}}
\end{Highlighting}
\end{Shaded}

\hypertarget{hamiltonian-monte-carlo}{%
\subsubsection{Hamiltonian Monte Carlo}\label{hamiltonian-monte-carlo}}

\textbf{Implement HMC (step-size and number of steps are parameters; you
can use unit mass matrix, but if you want to impress, you can tune the
diagonal elements as well)}

\begin{Shaded}
\begin{Highlighting}[]
\NormalTok{hmc }\OtherTok{\textless{}{-}} \ControlFlowTok{function}\NormalTok{(step\_size, n\_steps) \{}
  
\NormalTok{\}}
\end{Highlighting}
\end{Shaded}

\hypertarget{rejection-sample}{%
\subsubsection{Rejection sample}\label{rejection-sample}}

\textbf{Implement rejection sampling with your own choice of 2D
envelope. This needn't be done for (4), because the dimensionality is
too high.}

\begin{Shaded}
\begin{Highlighting}[]
\NormalTok{rejection\_sampling }\OtherTok{\textless{}{-}} \ControlFlowTok{function}\NormalTok{() \{}
  
\NormalTok{\}}
\end{Highlighting}
\end{Shaded}

For each of the scenarios: - Use each sampling algorithm twice - once
for some quick choice of MCMC paramters and once for tuned parameters.
You may tune by hand via trial and error or use some other approach. -
For each algorithm and run, generate 5 independent chains of 1000
samples. - Apply standard MCMC diagnostics for each algorithm/run
(traceplot for each parameter and all chains at the same time),
autocovariance, ESS, and ESS/second. - Compare the means of the samples
with the ground truth (for the bivariate normal and banana we know the
true means for x and y, for the logistic regression fit a
non-regularized regression with maximum-likelihood and compare sample
means with MLE parameters). - Discuss which algorithm is the most
successful in sampling from the target distribution. Include a
discussion of how difficult/easy it was to tune MCMC parameters.

\hypertarget{scenario-1-bivariate-standard-normal}{%
\subsection{Scenario 1: Bivariate standard
normal}\label{scenario-1-bivariate-standard-normal}}

\hypertarget{scenario-2-the-shape-of-the-banana-function}{%
\subsection{Scenario 2: The shape of the banana
function}\label{scenario-2-the-shape-of-the-banana-function}}

Distribution is in shape of the banana function from the examples.

\hypertarget{scenario-3-the-shape-of-the-logistic-regression-likelihood}{%
\subsection{Scenario 3: The shape of the logistic regression
likelihood}\label{scenario-3-the-shape-of-the-logistic-regression-likelihood}}

Distribution is in shape of the logistic regression likelihood for the
dataset provided in dataset.csv, but using only the first two ``x''
columns. Note that ``X1'' serves as the intercept, so don't have another
in the model.

For each of the above scenarios: - Use each sampling algorithm twice -
once for some quick choice of MCMC paramters and once for tuned
parameters. You may tune by hand via trial and error or use some other
approach. - For each algorithm and run, generate 5 independent chains of
1000 samples. - Apply standard MCMC diagnostics for each algorithm/run
(traceplot for each parameter and all chains at the same time),
autocovariance, ESS, and ESS/second. - Compare the means of the samples
with the ground truth (for the bivariate normal and banana we know the
true means for x and y, for the logistic regression fit a
non-regularized regression with maximum-likelihood and compare sample
means with MLE parameters). - Discuss which algorithm is the most
successful in sampling from the target distribution. Include a
discussion of how difficult/easy it was to tune MCMC parameters.

Notes: * For reference, the ground truth coefficients for the logistic
regression dataset are: 2.00 -0.84 -0.64 0.72 -0.10 -0.85 -0.97 0.23
-0.68 -0.42 0.47 However, the dataset is only a sample from this
process, so the estimates might diffrer. * HMC requires gradients! For
bivariate normal it is trivial (do them analytically), for the banana
they are already given in the R examples code, and for logistic
regression you can also do them by hand (all the parameters are
symmetrical, so it is just one partial derivative). * The key M-H and
HMC code is already available from the examples, but do study it and
make sure you understand how it relates to parts of the algorithms. Also
note that you only need one M-H and HMC implementation for all the
scenarios. Only the parameters, the target distribution, and its
gradient are different.

\end{document}
